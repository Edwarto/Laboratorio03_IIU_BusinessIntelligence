\documentclass{article}
\usepackage[utf8]{inputenc}

\title{Laboratorio01U2_INTELIGENCIA_NEGOCIOS}
\author{edwartbalcon}
\date{Septiembre 2021}

\usepackage[utf8]{inputenc}
\usepackage[spanish]{babel}
\usepackage{natbib}
\usepackage{graphicx}

\begin{document}

\title{Caratula}

\begin{titlepage}
\begin{center}
\begin{Large}
\textbf{UNIVERSIDAD PRIVADA DE TACNA} \\
\end{Large}
\vspace*{-0.025in}
\begin{figure}[htb]
\begin{center}
\includegraphics[width=6cm]{./images/logo_UPT}
\end{center}
\end{figure}
\vspace*{-0.025in}
\begin{Large}
\textbf{FACULTAD DE INGENIERIA} \\
\end{Large}
\vspace*{0.05in}
\begin{Large}
\textbf{Escuela Profesional de Ingeniería de Sistema} \\
\end{Large}


\vspace*{0.4in}

\vspace*{0.1in}
\begin{Large}
\textbf{Informe de laboratorio 03: Creando un Cubo Multimensional} \\
\end{Large}

\vspace*{0.3in}
\begin{Large}
\textbf{Curso: Inteligencia de negocios} \\
\end{Large}

\vspace*{0.3in}
\begin{Large}
\textbf{DOCENTE: Ing. Patrick Cuadros Quiroga} \\
\end{Large}

\vspace*{0.2in}
\vspace*{0.1in}
\begin{large}

\begin{Large}
\textbf{Alumno: Balcon Coahila, Edwart Juan\hfill	(2013046516) } \\
\end{Large}

\vspace*{0.15in}
\begin{Large}
\textbf{Tacna – Perú} \\
\end{Large}

\vspace*{0.05in}
\begin{Large}
\textbf{2021 } \\
\end{Large}

\end{large}
\end{center}

\end{titlepage}



%%INICIO Resumen
\section{Objetivos}
Crear un cubo Multidimensional, para lo cual se tiene que haber instalado antes el motor de Analysis
Services Multidimensional se necesita una base de datos para la creación del cubo, para lo que se
necesitaría tener restaurada la base de datos Adventure Works DW..
%%FIN Resumen




%%----------------------------------------------------------------------------------------------------------------------------------------------------------
%%INICIO Marco Teórico
\section{Procedimiento}

Abrir el SQL Server Data Tools y dirigirnos a la pestaña de Business Intelligence -> Analysis
Services. Como se creará un Modelo Multidimensional desde 0 , seleccionaremos la primera opción. En la
casilla de Name le colocamos un nombre al proyecto y a la solución:



\include{sections/Task01}

\include{sections/Task02}


\include{sections/Task03}


\include{sections/Task04}


\end{document}